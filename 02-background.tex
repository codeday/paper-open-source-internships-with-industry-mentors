\section{Internships and CS Education}

Most undergraduates pursuing a computer science degree chose the major because they believe it will improve their job prospects in industry, \cite{nortonPerceivedBenefitsUndergraduate2017,alshahraniUsingSocialCognitive2018,helpsStudentExpectationsComputing2005} but a disconnect between coursework and industry has long been reported by both graduates \cite{begelStrugglesNewCollege2008,craigListeningEarlyCareer2018,kapoorUnderstandingCSUndergraduate2019} and employers. \cite{begelStrugglesNewCollege2008} Institutions commonly try to resolve this disconnect through industry engagement opportunities such as capstone projects, mentoring, guest speakers, and internships.

Internships in particular provide benefits that are difficult to replicate in the classroom:

\begin{itemize}
    \item \textbf{Functioning on a team:}
    Software engineers must learn to deal with resistance from co-workers or managers or delays from others on tasks that block their progress and must likewise learn to prioritize and communicate their work to co-workers and managers. \cite{beaubouefComputerScienceCurriculum2011}
    
    \item \textbf{Career confidence:}
    Students who believe they have a path into a career put more time into educational activities and are more likely to overcome obstacles. The real-world practice, goal-setting, and performance feedback afforded by an internship can increase this confidence. \cite{lentUnifyingSocialCognitive1994} 

    \item \textbf{Recruiting and retention in the major:}
    Few students enroll in STEM majors, and many drop out. \cite{chenSTEMAttritionCollege2013} Prior studies have shown that the number and diversity of students entering and staying in these majors can be increased by providing opportunities for internships \cite{frylingCatchEmEarly2018} and research experience. \cite{dahlbergImprovingRetentionGraduate2008, tashakkoriEarlyParticipationCS2011}

    \item \textbf{Securing a Job After Graduation:}
    Studies have found that whether a student had completed an internship or not is one of the most significant variables as to whether or not they have a job after graduation, \cite{callananAssessingRoleInternships2004, jonesTransformingCurriculumPreparing2002, knouseRelationCollegeInternships1999, saltikoffPositiveImplicationsInternships2017} their starting salary, and the amount of time they spend looking for a job. \cite{gaultUndergraduateBusinessInternships2000} This relationship holds even when the internships are unpaid. \cite{saltikoffPositiveImplicationsInternships2017}
\end{itemize}

\subsection{Access to Internships}
For many students, internships are hard to come by. Among students seeking bachelor’s degrees, only 20\% of rising sophomores get internships, and less than half of rising juniors/seniors do. Even by graduation, only 60\% of students have internship experience. \cite{kapoorExploringParticipationCS2020, kocClass2014Student2014}

Access to internships is not equitable: students with a high household income are much more likely to get internships. One study found that, while most students with a household income over \$150,000 per year were offered an internship by the time they graduated, the rate dropped to only 35\% for students with a household income under \$100,000 per year. \cite{kapoorExploringParticipationCS2020} Although further studies are needed, it's likely that gender and race are additional factors, given the technology industry's reputation for being unwelcoming. The name recognition of the institution at which a student studies likely has a significant effect as well: indeed, implementation of this program was driven by the experiences of students attending small affordable schools, who expressed frustration that their schools were ignored by recruiters.